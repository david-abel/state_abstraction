\documentclass[11pt]{amsart}
\usepackage{geometry}                % See geometry.pdf to learn the layout options. There are lots.
\geometry{letterpaper}                   % ... or a4paper or a5paper or ... 
%% preamble.tex
%% this should be included with a command like
%% %% preamble.tex
%% this should be included with a command like
%% %% preamble.tex
%% this should be included with a command like
%% \input{p}

\usepackage{amsfonts}
\usepackage{amsthm}
\usepackage{latexsym}
\usepackage{amsmath}
\usepackage{amssymb}
\usepackage{latexsym}
\usepackage{graphicx}
\usepackage{wasysym}
\usepackage{hyperref}
\usepackage{mathtools}
\usepackage[makeroom]{cancel}
\usepackage{cleveref}
\usepackage{enumerate}
\usepackage{framed}
\usepackage{tikz}
\usepackage[framemethod=TikZ]{mdframed}
\usepackage{algorithm,algorithmic} % Add "[noend]" to get rid of loop/conditional "end" lines.
\usepackage{mathrsfs}

\newcommand{\sk}{s}
\newtheorem{theorem}{Theorem}
\newtheorem{lesson}{Lesson}
\newtheorem{proposition}{Proposition}
\newtheorem{lemma}{Lemma}
\newtheorem{corollary}{Corollary}
\newtheorem{fact}{Fact}
\newtheorem*{claim}{Claim}
\newtheorem{definition}{Definition}
\newtheorem{assumption}{Assumption}
\theoremstyle{remark}
\newtheorem{example}{Example}
\newtheorem*{remark}{Remark}

%==============================================================================
% Macros.
%==============================================================================
\newcommand{\new}[1]{{\em #1\/}}		% New term (set in italics).
\newcommand{\set}[1]{\{#1\}}			% Set (as in \set{1,2,3})
\newcommand{\setof}[2]{\{\,{#1}|~{#2}\,\}}	% Set (as in \setof{x}{x > 0})
\newcommand{\C}{\mathbb{C}}	                % Complex numbers.
\newcommand{\Q}{\textsf{Q}}                     % Robinson Arithmetic
\newcommand{\R}{\textsf{R}}                     % The system R
\newcommand{\PA}{\textsf{PA}}                     % Peano Arithmetic
\newcommand{\LA}{$\mathcal{LA}$}			% Language of Arithmetic
\newcommand{\compl}[1]{\overline{#1}}		% Complement of ...     
\newcommand{\subproblem}[1]{\vspace{4 mm}\noindent {\bf (#1)} \\} % Subproblem (a), (b), etc...
\newcommand{\elem}[1]{\noindent{\bf #1}}	    % Proof element (Claim:, Proof:, etc.)
\newcommand{\spacerule}{\vspace{4mm}\hrule\vspace{4mm}} % Add an hrule with some space
\newcommand{\dmod}[2]{\left[ #1\ \text{mod}\ #2\right]} % My mod rule
\newcommand{\tab}[1]{\hspace{.2\textwidth}\rlap{#1}}

\usepackage{eqparbox}
\renewcommand{\algorithmiccomment}[1]{\hfill\eqparbox{COMMENT}{\# #1}}

% Misc lines and other formatting
\newcommand{\encircle}[1]{\tikz[baseline=(char.base)]\node[anchor=south west, draw,rectangle, rounded corners, inner sep=2pt, minimum size=6mm, text height=2mm](char){#1} ;} % Circled text
\newcommand{\midline}{
\begin{center}
\noindent\rule{4cm}{0.4pt}
\end{center}}

\newcommand{\thm}[3]{
\begin{mdframed}[roundcorner=4pt, backgroundcolor=gray!5]
\vspace{1mm}
{\noindent{\bf #1 #2:}}{ #3}
\end{mdframed}
}

% GODEL NUMBERS
\newbox\gnBoxA
\newdimen\gnCornerHgt
\setbox\gnBoxA=\hbox{$\ulcorner$}
\global\gnCornerHgt=\ht\gnBoxA
\newdimen\gnArgHgt
\def\Godelnum #1{%
\setbox\gnBoxA=\hbox{$#1$}%
\gnArgHgt=\ht\gnBoxA%
\ifnum     \gnArgHgt<\gnCornerHgt \gnArgHgt=0pt%
\else \advance \gnArgHgt by -\gnCornerHgt%
\fi \raise\gnArgHgt\hbox{$\ulcorner$} \box\gnBoxA %
\raise\gnArgHgt\hbox{$\urcorner$}}

\DeclarePairedDelimiter{\ceil}{\lceil}{\rceil}
\DeclareSymbolFont{AMSb}{U}{msb}{m}{n}
\DeclareMathSymbol{\F}{\mathalpha}{AMSb}{"46}
\DeclareMathSymbol{\N}{\mathalpha}{AMSb}{"4E}
\DeclareMathSymbol{\X}{\mathalpha}{AMSb}{"58}
\DeclareMathSymbol{\Zz}{\mathalpha}{AMSb}{"5A}
\DeclareMathOperator*{\argmin}{arg\,min}
\DeclareMathOperator*{\argmax}{arg\,max}
\newcommand{\Z}[1]{{\ensuremath{\Zz_{#1}} }}
\newcommand{\Zs}[1]{\ensuremath{\Zz^{\ast}_{#1}}}
\newcommand{\Zn}{\Z{n}}
\newcommand{\Zns}{\Zs{n}}
\newcommand{\Zstar}[1]{\Zs{#1}}
\newcommand{\Zp}{{\Z{p}}}
\newcommand{\Zps}{\Zs{p}}
\newcommand{\Zqs}{\Zs{q}}
\newcommand{\Zq}{{\Z{q}}}
\newcommand{\ord}[1]{\mathop{\mathrm{ord}}({#1})}
\newcommand{\QR}[1]{\ensuremath{\textit{QR}_{#1}}}
\newcommand{\becomes}{:=}
\newcommand{\rem}[1]{\ensuremath{\ \operatorname{rem} #1}}  
\newcommand{\U}{{\mathcal{U}}}
\newcommand{\floor}[1]{\ensuremath{\lfloor{#1}\rfloor}}
\newcommand{\de}[1]{\ensuremath{\Delta{#1}}}
\newcommand{\js}[2]{\left( \frac{#1}{#2} \right)}
\newcommand{\E}[1]{{\bf E} \left[ #1 \right]}
\newcommand{\PR}[1]{{\bf Pr} \left[ #1 \right]}

\renewcommand{\QR}{{\mbox{QR}}}
\newcommand{\QNR}{{\mbox{QNR}}}
\newcommand{\crt}{{\mbox{CRT}}}
\newcommand{\rsa}{{\mbox{RSA}}}
\newcommand{\rsamod}{{\mbox{RSA-modulus}}}


\newcommand{\greq}[1]{\stackrel{#1}{=}}
\newcommand{\hash}{\ensuremath{\mathcal{H}}}
\newcommand{\negl}{{\tt neg}}
\newcommand{\cindist}{\stackrel{c}{\approx}}
\newcommand{\A}{{\mathcal{A}}}
\newcommand{\B}{{\mathcal{B}}}

\newcommand{\gen}{\mathit{Gen}}
\newcommand{\keygen}{\mathit{KeyGen}}
\newcommand{\RSA}{\mathit{RSA}}
\newcommand{\pk}{\mathit{pk}}
\newcommand{\PK}{\mathit{PK}}
\newcommand{\SK}{\mathit{SK}}
\newcommand{\sign}{\mathit{Sign}}
\newcommand{\verify}{\mathit{Verify}}

% Bew command
\newcommand{\Bew}[1]{\textsc{Bew}(\Godelnum{#1})}
\newcommand{\Prf}[1]{\textsc{Prf}#1}
\newcommand{\udot}[1]{\underset{\centerdot}{#1}}


\setlength{\oddsidemargin}{.25in}
\setlength{\evensidemargin}{.25in}
\setlength{\textwidth}{6in}
\setlength{\topmargin}{-0.4in}
\setlength{\textheight}{8.5in}

\newcommand{\handout}[5]{
   \renewcommand{\thepage}{#1-\arabic{page}}
   \noindent
   \begin{center}
   \framebox{
      \vbox{
    \hbox to 5.78in { {\bf PHIL1885: Incompleteness} \hfill #2 }
       \vspace{4mm}
       \hbox to 5.78in { {\Large \hfill #5  \hfill} }
       \vspace{2mm}
       \hbox to 5.78in { {\it #3 \hfill #4} }
      }
   }
   \end{center}
   \vspace*{4mm}
}

% REMOVES ALL INDENTATION
% \setlength{\parindent}{0pt}

\usepackage{parskip}

\newcommand{\ho}[4]{\handout{#1}{#2}{Instructor:
#3}{}{Handout #1: #4}}

\newcommand{\lnotes}[3]{\handout{#1}{#2}{Instructor:
#3}{}{Lecture #1}}

\newcommand{\solution}[1]{#1}
%\homework{number}{out}{due}{instructor}
\newcommand{\homework}[4]{\handout{HW #1}{#2}{Instructor: #4}{Due: #3}{Homework #1}}

%================================================
% problemset macros
%================================================
% count problems
%\newcounter{solutioncount}
%\setcounter{solutioncount}{0}
\newcommand{\problem}[1]{%
%\addtocounter{solutioncount}{1}%
\section*{Problem #1:}}

% lets you make alphabetical lists (at first-level of enumeration)
\newenvironment
  {alphabetize}{\renewcommand{\theenumi}{\alph{enumi}}\begin{enumerate}}
  {\end{enumerate}\renewcommand{\theenumi}{\arabic{enumi}}}

\usepackage{amsfonts}
\usepackage{amsthm}
\usepackage{latexsym}
\usepackage{amsmath}
\usepackage{amssymb}
\usepackage{latexsym}
\usepackage{graphicx}
\usepackage{wasysym}
\usepackage{hyperref}
\usepackage{mathtools}
\usepackage[makeroom]{cancel}
\usepackage{cleveref}
\usepackage{enumerate}
\usepackage{framed}
\usepackage{tikz}
\usepackage[framemethod=TikZ]{mdframed}
\usepackage{algorithm,algorithmic} % Add "[noend]" to get rid of loop/conditional "end" lines.
\usepackage{mathrsfs}

\newcommand{\sk}{s}
\newtheorem{theorem}{Theorem}
\newtheorem{lesson}{Lesson}
\newtheorem{proposition}{Proposition}
\newtheorem{lemma}{Lemma}
\newtheorem{corollary}{Corollary}
\newtheorem{fact}{Fact}
\newtheorem*{claim}{Claim}
\newtheorem{definition}{Definition}
\newtheorem{assumption}{Assumption}
\theoremstyle{remark}
\newtheorem{example}{Example}
\newtheorem*{remark}{Remark}

%==============================================================================
% Macros.
%==============================================================================
\newcommand{\new}[1]{{\em #1\/}}		% New term (set in italics).
\newcommand{\set}[1]{\{#1\}}			% Set (as in \set{1,2,3})
\newcommand{\setof}[2]{\{\,{#1}|~{#2}\,\}}	% Set (as in \setof{x}{x > 0})
\newcommand{\C}{\mathbb{C}}	                % Complex numbers.
\newcommand{\Q}{\textsf{Q}}                     % Robinson Arithmetic
\newcommand{\R}{\textsf{R}}                     % The system R
\newcommand{\PA}{\textsf{PA}}                     % Peano Arithmetic
\newcommand{\LA}{$\mathcal{LA}$}			% Language of Arithmetic
\newcommand{\compl}[1]{\overline{#1}}		% Complement of ...     
\newcommand{\subproblem}[1]{\vspace{4 mm}\noindent {\bf (#1)} \\} % Subproblem (a), (b), etc...
\newcommand{\elem}[1]{\noindent{\bf #1}}	    % Proof element (Claim:, Proof:, etc.)
\newcommand{\spacerule}{\vspace{4mm}\hrule\vspace{4mm}} % Add an hrule with some space
\newcommand{\dmod}[2]{\left[ #1\ \text{mod}\ #2\right]} % My mod rule
\newcommand{\tab}[1]{\hspace{.2\textwidth}\rlap{#1}}

\usepackage{eqparbox}
\renewcommand{\algorithmiccomment}[1]{\hfill\eqparbox{COMMENT}{\# #1}}

% Misc lines and other formatting
\newcommand{\encircle}[1]{\tikz[baseline=(char.base)]\node[anchor=south west, draw,rectangle, rounded corners, inner sep=2pt, minimum size=6mm, text height=2mm](char){#1} ;} % Circled text
\newcommand{\midline}{
\begin{center}
\noindent\rule{4cm}{0.4pt}
\end{center}}

\newcommand{\thm}[3]{
\begin{mdframed}[roundcorner=4pt, backgroundcolor=gray!5]
\vspace{1mm}
{\noindent{\bf #1 #2:}}{ #3}
\end{mdframed}
}

% GODEL NUMBERS
\newbox\gnBoxA
\newdimen\gnCornerHgt
\setbox\gnBoxA=\hbox{$\ulcorner$}
\global\gnCornerHgt=\ht\gnBoxA
\newdimen\gnArgHgt
\def\Godelnum #1{%
\setbox\gnBoxA=\hbox{$#1$}%
\gnArgHgt=\ht\gnBoxA%
\ifnum     \gnArgHgt<\gnCornerHgt \gnArgHgt=0pt%
\else \advance \gnArgHgt by -\gnCornerHgt%
\fi \raise\gnArgHgt\hbox{$\ulcorner$} \box\gnBoxA %
\raise\gnArgHgt\hbox{$\urcorner$}}

\DeclarePairedDelimiter{\ceil}{\lceil}{\rceil}
\DeclareSymbolFont{AMSb}{U}{msb}{m}{n}
\DeclareMathSymbol{\F}{\mathalpha}{AMSb}{"46}
\DeclareMathSymbol{\N}{\mathalpha}{AMSb}{"4E}
\DeclareMathSymbol{\X}{\mathalpha}{AMSb}{"58}
\DeclareMathSymbol{\Zz}{\mathalpha}{AMSb}{"5A}
\DeclareMathOperator*{\argmin}{arg\,min}
\DeclareMathOperator*{\argmax}{arg\,max}
\newcommand{\Z}[1]{{\ensuremath{\Zz_{#1}} }}
\newcommand{\Zs}[1]{\ensuremath{\Zz^{\ast}_{#1}}}
\newcommand{\Zn}{\Z{n}}
\newcommand{\Zns}{\Zs{n}}
\newcommand{\Zstar}[1]{\Zs{#1}}
\newcommand{\Zp}{{\Z{p}}}
\newcommand{\Zps}{\Zs{p}}
\newcommand{\Zqs}{\Zs{q}}
\newcommand{\Zq}{{\Z{q}}}
\newcommand{\ord}[1]{\mathop{\mathrm{ord}}({#1})}
\newcommand{\QR}[1]{\ensuremath{\textit{QR}_{#1}}}
\newcommand{\becomes}{:=}
\newcommand{\rem}[1]{\ensuremath{\ \operatorname{rem} #1}}  
\newcommand{\U}{{\mathcal{U}}}
\newcommand{\floor}[1]{\ensuremath{\lfloor{#1}\rfloor}}
\newcommand{\de}[1]{\ensuremath{\Delta{#1}}}
\newcommand{\js}[2]{\left( \frac{#1}{#2} \right)}
\newcommand{\E}[1]{{\bf E} \left[ #1 \right]}
\newcommand{\PR}[1]{{\bf Pr} \left[ #1 \right]}

\renewcommand{\QR}{{\mbox{QR}}}
\newcommand{\QNR}{{\mbox{QNR}}}
\newcommand{\crt}{{\mbox{CRT}}}
\newcommand{\rsa}{{\mbox{RSA}}}
\newcommand{\rsamod}{{\mbox{RSA-modulus}}}


\newcommand{\greq}[1]{\stackrel{#1}{=}}
\newcommand{\hash}{\ensuremath{\mathcal{H}}}
\newcommand{\negl}{{\tt neg}}
\newcommand{\cindist}{\stackrel{c}{\approx}}
\newcommand{\A}{{\mathcal{A}}}
\newcommand{\B}{{\mathcal{B}}}

\newcommand{\gen}{\mathit{Gen}}
\newcommand{\keygen}{\mathit{KeyGen}}
\newcommand{\RSA}{\mathit{RSA}}
\newcommand{\pk}{\mathit{pk}}
\newcommand{\PK}{\mathit{PK}}
\newcommand{\SK}{\mathit{SK}}
\newcommand{\sign}{\mathit{Sign}}
\newcommand{\verify}{\mathit{Verify}}

% Bew command
\newcommand{\Bew}[1]{\textsc{Bew}(\Godelnum{#1})}
\newcommand{\Prf}[1]{\textsc{Prf}#1}
\newcommand{\udot}[1]{\underset{\centerdot}{#1}}


\setlength{\oddsidemargin}{.25in}
\setlength{\evensidemargin}{.25in}
\setlength{\textwidth}{6in}
\setlength{\topmargin}{-0.4in}
\setlength{\textheight}{8.5in}

\newcommand{\handout}[5]{
   \renewcommand{\thepage}{#1-\arabic{page}}
   \noindent
   \begin{center}
   \framebox{
      \vbox{
    \hbox to 5.78in { {\bf PHIL1885: Incompleteness} \hfill #2 }
       \vspace{4mm}
       \hbox to 5.78in { {\Large \hfill #5  \hfill} }
       \vspace{2mm}
       \hbox to 5.78in { {\it #3 \hfill #4} }
      }
   }
   \end{center}
   \vspace*{4mm}
}

% REMOVES ALL INDENTATION
% \setlength{\parindent}{0pt}

\usepackage{parskip}

\newcommand{\ho}[4]{\handout{#1}{#2}{Instructor:
#3}{}{Handout #1: #4}}

\newcommand{\lnotes}[3]{\handout{#1}{#2}{Instructor:
#3}{}{Lecture #1}}

\newcommand{\solution}[1]{#1}
%\homework{number}{out}{due}{instructor}
\newcommand{\homework}[4]{\handout{HW #1}{#2}{Instructor: #4}{Due: #3}{Homework #1}}

%================================================
% problemset macros
%================================================
% count problems
%\newcounter{solutioncount}
%\setcounter{solutioncount}{0}
\newcommand{\problem}[1]{%
%\addtocounter{solutioncount}{1}%
\section*{Problem #1:}}

% lets you make alphabetical lists (at first-level of enumeration)
\newenvironment
  {alphabetize}{\renewcommand{\theenumi}{\alph{enumi}}\begin{enumerate}}
  {\end{enumerate}\renewcommand{\theenumi}{\arabic{enumi}}}

\usepackage{amsfonts}
\usepackage{amsthm}
\usepackage{latexsym}
\usepackage{amsmath}
\usepackage{amssymb}
\usepackage{latexsym}
\usepackage{graphicx}
\usepackage{wasysym}
\usepackage{hyperref}
\usepackage{mathtools}
\usepackage[makeroom]{cancel}
\usepackage{cleveref}
\usepackage{enumerate}
\usepackage{framed}
\usepackage{tikz}
\usepackage[framemethod=TikZ]{mdframed}
\usepackage{algorithm,algorithmic} % Add "[noend]" to get rid of loop/conditional "end" lines.
\usepackage{mathrsfs}

\newcommand{\sk}{s}
\newtheorem{theorem}{Theorem}
\newtheorem{lesson}{Lesson}
\newtheorem{proposition}{Proposition}
\newtheorem{lemma}{Lemma}
\newtheorem{corollary}{Corollary}
\newtheorem{fact}{Fact}
\newtheorem*{claim}{Claim}
\newtheorem{definition}{Definition}
\newtheorem{assumption}{Assumption}
\theoremstyle{remark}
\newtheorem{example}{Example}
\newtheorem*{remark}{Remark}

%==============================================================================
% Macros.
%==============================================================================
\newcommand{\new}[1]{{\em #1\/}}		% New term (set in italics).
\newcommand{\set}[1]{\{#1\}}			% Set (as in \set{1,2,3})
\newcommand{\setof}[2]{\{\,{#1}|~{#2}\,\}}	% Set (as in \setof{x}{x > 0})
\newcommand{\C}{\mathbb{C}}	                % Complex numbers.
\newcommand{\Q}{\textsf{Q}}                     % Robinson Arithmetic
\newcommand{\R}{\textsf{R}}                     % The system R
\newcommand{\PA}{\textsf{PA}}                     % Peano Arithmetic
\newcommand{\LA}{$\mathcal{LA}$}			% Language of Arithmetic
\newcommand{\compl}[1]{\overline{#1}}		% Complement of ...     
\newcommand{\subproblem}[1]{\vspace{4 mm}\noindent {\bf (#1)} \\} % Subproblem (a), (b), etc...
\newcommand{\elem}[1]{\noindent{\bf #1}}	    % Proof element (Claim:, Proof:, etc.)
\newcommand{\spacerule}{\vspace{4mm}\hrule\vspace{4mm}} % Add an hrule with some space
\newcommand{\dmod}[2]{\left[ #1\ \text{mod}\ #2\right]} % My mod rule
\newcommand{\tab}[1]{\hspace{.2\textwidth}\rlap{#1}}

\usepackage{eqparbox}
\renewcommand{\algorithmiccomment}[1]{\hfill\eqparbox{COMMENT}{\# #1}}

% Misc lines and other formatting
\newcommand{\encircle}[1]{\tikz[baseline=(char.base)]\node[anchor=south west, draw,rectangle, rounded corners, inner sep=2pt, minimum size=6mm, text height=2mm](char){#1} ;} % Circled text
\newcommand{\midline}{
\begin{center}
\noindent\rule{4cm}{0.4pt}
\end{center}}

\newcommand{\thm}[3]{
\begin{mdframed}[roundcorner=4pt, backgroundcolor=gray!5]
\vspace{1mm}
{\noindent{\bf #1 #2:}}{ #3}
\end{mdframed}
}

% GODEL NUMBERS
\newbox\gnBoxA
\newdimen\gnCornerHgt
\setbox\gnBoxA=\hbox{$\ulcorner$}
\global\gnCornerHgt=\ht\gnBoxA
\newdimen\gnArgHgt
\def\Godelnum #1{%
\setbox\gnBoxA=\hbox{$#1$}%
\gnArgHgt=\ht\gnBoxA%
\ifnum     \gnArgHgt<\gnCornerHgt \gnArgHgt=0pt%
\else \advance \gnArgHgt by -\gnCornerHgt%
\fi \raise\gnArgHgt\hbox{$\ulcorner$} \box\gnBoxA %
\raise\gnArgHgt\hbox{$\urcorner$}}

\DeclarePairedDelimiter{\ceil}{\lceil}{\rceil}
\DeclareSymbolFont{AMSb}{U}{msb}{m}{n}
\DeclareMathSymbol{\F}{\mathalpha}{AMSb}{"46}
\DeclareMathSymbol{\N}{\mathalpha}{AMSb}{"4E}
\DeclareMathSymbol{\X}{\mathalpha}{AMSb}{"58}
\DeclareMathSymbol{\Zz}{\mathalpha}{AMSb}{"5A}
\DeclareMathOperator*{\argmin}{arg\,min}
\DeclareMathOperator*{\argmax}{arg\,max}
\newcommand{\Z}[1]{{\ensuremath{\Zz_{#1}} }}
\newcommand{\Zs}[1]{\ensuremath{\Zz^{\ast}_{#1}}}
\newcommand{\Zn}{\Z{n}}
\newcommand{\Zns}{\Zs{n}}
\newcommand{\Zstar}[1]{\Zs{#1}}
\newcommand{\Zp}{{\Z{p}}}
\newcommand{\Zps}{\Zs{p}}
\newcommand{\Zqs}{\Zs{q}}
\newcommand{\Zq}{{\Z{q}}}
\newcommand{\ord}[1]{\mathop{\mathrm{ord}}({#1})}
\newcommand{\QR}[1]{\ensuremath{\textit{QR}_{#1}}}
\newcommand{\becomes}{:=}
\newcommand{\rem}[1]{\ensuremath{\ \operatorname{rem} #1}}  
\newcommand{\U}{{\mathcal{U}}}
\newcommand{\floor}[1]{\ensuremath{\lfloor{#1}\rfloor}}
\newcommand{\de}[1]{\ensuremath{\Delta{#1}}}
\newcommand{\js}[2]{\left( \frac{#1}{#2} \right)}
\newcommand{\E}[1]{{\bf E} \left[ #1 \right]}
\newcommand{\PR}[1]{{\bf Pr} \left[ #1 \right]}

\renewcommand{\QR}{{\mbox{QR}}}
\newcommand{\QNR}{{\mbox{QNR}}}
\newcommand{\crt}{{\mbox{CRT}}}
\newcommand{\rsa}{{\mbox{RSA}}}
\newcommand{\rsamod}{{\mbox{RSA-modulus}}}


\newcommand{\greq}[1]{\stackrel{#1}{=}}
\newcommand{\hash}{\ensuremath{\mathcal{H}}}
\newcommand{\negl}{{\tt neg}}
\newcommand{\cindist}{\stackrel{c}{\approx}}
\newcommand{\A}{{\mathcal{A}}}
\newcommand{\B}{{\mathcal{B}}}

\newcommand{\gen}{\mathit{Gen}}
\newcommand{\keygen}{\mathit{KeyGen}}
\newcommand{\RSA}{\mathit{RSA}}
\newcommand{\pk}{\mathit{pk}}
\newcommand{\PK}{\mathit{PK}}
\newcommand{\SK}{\mathit{SK}}
\newcommand{\sign}{\mathit{Sign}}
\newcommand{\verify}{\mathit{Verify}}

% Bew command
\newcommand{\Bew}[1]{\textsc{Bew}(\Godelnum{#1})}
\newcommand{\Prf}[1]{\textsc{Prf}#1}
\newcommand{\udot}[1]{\underset{\centerdot}{#1}}


\setlength{\oddsidemargin}{.25in}
\setlength{\evensidemargin}{.25in}
\setlength{\textwidth}{6in}
\setlength{\topmargin}{-0.4in}
\setlength{\textheight}{8.5in}

\newcommand{\handout}[5]{
   \renewcommand{\thepage}{#1-\arabic{page}}
   \noindent
   \begin{center}
   \framebox{
      \vbox{
    \hbox to 5.78in { {\bf PHIL1885: Incompleteness} \hfill #2 }
       \vspace{4mm}
       \hbox to 5.78in { {\Large \hfill #5  \hfill} }
       \vspace{2mm}
       \hbox to 5.78in { {\it #3 \hfill #4} }
      }
   }
   \end{center}
   \vspace*{4mm}
}

% REMOVES ALL INDENTATION
% \setlength{\parindent}{0pt}

\usepackage{parskip}

\newcommand{\ho}[4]{\handout{#1}{#2}{Instructor:
#3}{}{Handout #1: #4}}

\newcommand{\lnotes}[3]{\handout{#1}{#2}{Instructor:
#3}{}{Lecture #1}}

\newcommand{\solution}[1]{#1}
%\homework{number}{out}{due}{instructor}
\newcommand{\homework}[4]{\handout{HW #1}{#2}{Instructor: #4}{Due: #3}{Homework #1}}

%================================================
% problemset macros
%================================================
% count problems
%\newcounter{solutioncount}
%\setcounter{solutioncount}{0}
\newcommand{\problem}[1]{%
%\addtocounter{solutioncount}{1}%
\section*{Problem #1:}}

% lets you make alphabetical lists (at first-level of enumeration)
\newenvironment
  {alphabetize}{\renewcommand{\theenumi}{\alph{enumi}}\begin{enumerate}}
  {\end{enumerate}\renewcommand{\theenumi}{\arabic{enumi}}}
\title{Approximating the Optimal State Abstraction}
\author{}
\date{}                                           % Activate to display a given date or no date

% --- Note Commands ---
\newcommand\davenote[1]{\textcolor{blue}{Dave: #1}}

\begin{document}
\maketitle
\section{Results}

Recall, there is a ground MDP $M$ we want to solve. We homogenize to an MDP $M'$, and apply $\phi$ to $M'$ to get an abstract MDP $A$.

For each approximate $\phi$ type, we want to show, $\forall_{\epsilon} : \epsilon \in [0 : \infty)$
\begin{equation}
\forall_{s,a} : | V^{\pi^*_M}(s) - V^{\pi^*_{M'}}(s) | \leq \epsilon
\end{equation}

We consider four $\phi$ types:
% phi model
\begin{align*}
\encircle{\text{ $\phi_m$ }}\ & : \phi_{m}(s_1) = \phi_m(s_2) \rightarrow \forall_a |\mathcal{R}(s_1,a) - \mathcal{R}(s_2,a)| \leq \epsilon \ \wedge\ 
\forall_{a,s' \in \phi^{-1}(\bar{s})} | \mathcal{T}(s_1,a,s') - \mathcal{T}(s_2,a,s') | \leq \epsilon \\
\encircle{\text{ $\phi_{Q^*}$ }}\ & : \phi_{Q^*}(s_1) = \phi_{Q^*}(s_2) \rightarrow \forall_a |Q^*(s_1,a) - Q^*(s_2,a)| \leq \epsilon \\
\encircle{\text{ $\phi_{D}$ }}\ & : \phi_D(s_1) = \phi_D(s_2) \rightarrow \Delta \left( \Pr_{a \in \mathcal{A}}(a \mid s_1) \mid\mid \Pr_{a \in \mathcal{A}}(a \mid s_2) \right) \leq \epsilon \\
\encircle{\text{ $\phi_{a^*}$ }}\ & : \phi_{a^*}(s_1) = \phi_{a^*}(s_2) \rightarrow \argmax_a Q^*(s_1,a) = \argmax_a Q^*(s_2, a) \wedge |\max_a Q^*(s_1,a) - \max_a Q^*(s_2,a)| \leq \epsilon  \\
\end{align*}

\elem{NOTE:} Instead of performing the abstraction w.r.t. $\phi$, our new strategy is to, given an MDP $M$ (and possibly some other information, $\kappa$), apply a function $H$ to create $M'$, for which all of the old $\phi$ functions may operate.

Specifically:
\begin{multline}
H(f, \epsilon) \rightarrow f' \text{  s.t.  } \\ \forall_{(s_1,a,s_1'),(s_2,a,s_2')} \left(f'(s_1,a,s_1') = f'(s_2,a,s_2') \rightarrow | f(s_1,a,s_1') - f(s_2,a,s_2') | \leq \epsilon\right)
\end{multline}

Where:
\begin{equation}
f(s,a,s') \rightarrow \mathbb{R}
\end{equation}

% Other results to consider: phi compression can be arbitrarily bad for other methods of approximate compression. Sample complexity increase can be arbitrarily large.

\elem{NOTE:} We are {\it not} going to provide results for $\phi_{\pi^*}$, since relaxing equality of optimal actions doesn't mean anything.

\elem{NOTE:} Additionally, we are {\it not} going to provide results for $\phi_{Q^\pi}$, since as Lihong's paper notes, ``It is an open question how to find $Q^\pi$ irrelevance abstractions without enumerating all possible policies, they do not give results. Furthermore, an MDP for which this is true is an awfully weird MDP...

\elem{NOTE:} For $\phi_D$, we consider the space of distributions on optimal Q values where:
\begin{equation}
f(Q(s,a)) > f(Q(s,a_2)) \rightarrow Q(s,a_1) > Q(s,a_2)
\end{equation}

\newpage
% Subsection: phi_m Results
\section{$\phi_{m}$ Results}

Let the maximum upper bound between optimal behavior in the noisy and ground MDP be called $B_s$.

\begin{align*}
B_{s} &= \max_{s,a}|Q'(s,a)-Q(s,a)|\\
B_{s} &= |R_s' - R_s + \gamma \sum_{s'}\left[T'Q'-TQ\right]|\\
&=|R'_s-R_s + \gamma \sum_{s'}\left[(T+T'-T)Q'-TQ\right]|\\
&\leq |R'_s-R_s|+|\gamma \sum_{s'}\left[(T+\epsilon_{\text{transition}})Q' - TQ\right]|\\
&\leq \epsilon_{\text{reward}}+|\gamma \sum_{s'}\left[(TQ'+\epsilon_{\text{transition}}Q' - TQ\right]|\\
&\leq \epsilon_{\text{reward}}+|\gamma \sum_{s'}\left[(T(Q'-Q)+\epsilon_{\text{transition}}Q' \right]|\\
&\leq \epsilon_{\text{reward}}+|\gamma |\mathcal{S}|\epsilon_{transition}Q'_{\text{max}} + \gamma \sum_{s'}\left[ T B_s\right]|\\
&\leq \epsilon_{\text{reward}} + \gamma B_{s} + \gamma |\mathcal{S}|\epsilon_{\text{transition}}Q'_{max}|\\
B_{s} &\leq 
\frac
{{\epsilon_{\text{reward}}} + \gamma |\mathcal{S}|\epsilon_{transition}V'_{\text{max}}}
{1-\gamma}
\end{align*}

Therefore, since there is a bound on $B_s$, we can conclude that:
\begin{equation}
\forall_{s,a} : |Q'(s,a) - Q(s,a) \leq \frac{{\epsilon_{\text{reward}}} + \gamma |\mathcal{S}|\epsilon_{transition}V'_{\text{max}}}{1-\gamma}
\end{equation}

Consequently, from the below $\phi_Q^*$ results, we conclude that:
\begin{equation}
\forall_s : |V^{\pi^*_M}(s) - V^{\pi^*_{M'}}(s)| \leq \frac{{\epsilon_{\text{reward}}} + \gamma |\mathcal{S}|\epsilon_{transition}V'_{\text{max}}}{1-\gamma}
\end{equation}
\qed


\newpage
% Subsection: phi_Q^* Results
\section{$\phi_{Q^*}$ Results}

We know:
\begin{equation}
\forall_{s,a} : | Q^{\pi^*_M}(s,a) - Q^{\pi^*_{M'}}(s,a)| \leq \epsilon
\label{eq:phi_q_assumption}
\end{equation}

Since `max' is a non-expansion, we know:
\begin{equation}
\forall_s : |\max_a Q^{\pi^*_M}(s,a) - \max_{a'} Q^{\pi^*_{M'}}(s,a')| \leq \max_b | Q^{\pi^*_M}(s,b) - Q^{\pi^*_{M'}}(s,b)|
\end{equation}

From Equation~\ref{eq:phi_q_assumption}, we know:
\begin{equation}
\forall_s : |\max_a Q^{\pi^*_M}(s,a) - \max_{a'} Q^{\pi^*_{M'}}(s,a')| \leq \max_b | Q^{\pi^*_M}(s,b) - Q^{\pi^*_{M'}}(s,b)| \leq \epsilon
\end{equation}

But since $\forall_\pi : V^{\pi} = \max_a Q^\pi(s,a)$, we conclude:
\begin{equation}
\forall_s : |V^{\pi^*_M}(s) - V^{\pi^*_{M'}}(s)| \leq \epsilon
\end{equation}
\qed

\newpage
% Subsection: phi_D Results
\section{$\phi_{D}$ Results}

\begin{equation}
\forall_{s_1, s_2 \in \mathcal{S}_M} : \phi_D(s_1) = \phi_D(s_2) \rightarrow \forall_{i \in [0:|\mathcal{A}| - 1]} | g_i(Q_M^*(s_1, \cdot) - g_i(Q_M^*(s_2, \cdot) | \leq \epsilon
\label{eq:phi_d}
\end{equation}

Where:
\begin{equation}
g(Q_M^*(s,\cdot)) : \mathbb{R}^{|\mathcal{A}|} \longmapsto \mathbb{R}^{|\mathcal{A}|}
\end{equation}

And there is a bound on the derivative of $g$:
\begin{equation}
\forall_{i \in [0:\mathcal{A}-1]} : K \cdot |g_i(\vec{x}) - g_i(\vec{y})| \geq |x_i - y_i |
\label{eq:bounded_derivative}
\end{equation}

From Equation~\ref{eq:phi_d} we know:
\begin{equation}
|g_i(Q_M^*(s_1,\cdot)) - g_i(Q_M^*(s_2,\cdot))| \leq \epsilon
\end{equation}

From Equation~\ref{eq:bounded_derivative} we know:
\begin{equation}
|Q_i(s_1,\cdot) - Q_i(s_2,\cdot)| \leq K |g_i(Q_M^*(s_1,\cdot)) - g_i(Q_M^*(s_2,\cdot))|
\end{equation}

Therefore:
\begin{equation}
|Q_i(s_1,\cdot) - Q_i(s_2,\cdot)| \leq K \epsilon
\end{equation}

From our previous $Q^*$ results, we conclude \qed




%Instead of letting $\phi_D$ denote a probability distribution\footnote{There are issue with this. Effectively, once the states are collapsed, there are arbitrarily many actual Q functions that could have given rise to the same action distributions, so we can't really say much in M' about Q.} we instead consider a more general class of abstraction functions:
%\begin{equation}
%\tau : \forall_{s_1,s_2 \in \mathcal{S}_M} : \tau(s_1) = \tau(s_2) \rightarrow \forall_a : | \Omega(s_1,a) - \Omega(s_2,a) | \leq \epsilon
%\end{equation}
%
%That is, we consider abstraction where some function $\Omega : \mathcal{S} \times \mathcal{A} \rightarrow \mathbb{R}$ is bounded for two states
%
%We want to prove:
%\begin{equation}
%\forall_{s \in \mathcal{S}_M} | V^{\pi_M^*}(s) - V^{\pi_A^*}(s) | \leq f(\epsilon)
%\end{equation}
%
%Since:
%\begin{equation}
%V^{\pi_A^*}(s) = Q^{\pi_A^*}(s, \pi_A^*(\tau(s)))
%\end{equation}
%
%We can rewrite in terms of $Q$ functions:
%\begin{equation}
%\forall_{s \in \mathcal{S}_M} : | Q^{\pi_M^*}(s, \pi_M^*(s)) - Q^{\pi_A^*}(s, \pi_A^*(\tau(s))) | \leq f(\epsilon)
%\label{eq:phi_d_q_func}
%\end{equation}
%
%Now, for any states we collapsed, we know:
%\begin{equation}
%\forall_{a \in \mathcal{A}} : | \Omega(s_1,a) - \Omega(s_2,a) | \leq \epsilon
%\label{eq:omega_consequent}
%\end{equation}
%
%In this setup, we can place constraints on $\Omega$ that will let us get from Equation~\ref{eq:omega_consequent} to Equation~\ref{eq:phi_d_q_func}
%
%Based on our earlier discussions, there seem to be three things that make sense:
%\begin{align*}
%\Omega(s,a) &= g(\mathcal{T}(s,a)) \\
%\Omega(s,a) &= g(\mathcal{R}(s,a)) \\
%\Omega^\pi(s,a) &= g(Q^\pi(s,a))
%\end{align*}
%For some function $g$.
%
%Suppose we go with the third. What properties need to be true of $g$ for this to go through?
%\begin{enumerate}[A.]
%\item $g$ is linear
%\item $g$ is monotonically increasing
%\item For a given domain of acceptable $Q^\pi$ values, $g$ has an infimum and supremum
%\end{enumerate}
%
%Then, from Equation~\ref{eq:omega_consequent}, we know:
%\begin{equation}
%|g(Q^{\pi_M^*}(s,\pi_M^*(s))) - g(Q^{\pi_A^*}(s, \pi_A^*(\tau(s)))) | \leq \epsilon
%\end{equation}
%
%Since $g$ is linear:
%\begin{equation}
%|g\left(Q^{\pi_M^*}\left(s,\pi_M^*(s)\right) - Q^{\pi_A^*}\left(s, \pi_A^*(\tau(s))\right)\right) | \leq \epsilon
%\end{equation}
%\begin{equation}
%\therefore g\left(Q^{\pi_M^*}\left(s,\pi_M^*(s)\right) - Q^{\pi_A^*}\left(s, \pi_A^*(\tau(s))\right)\right) \leq \epsilon
%\end{equation}
%
%What we want to say here is that by properties (2) and (3), since it's monotonically increasing, and there is a supremum and infimum, we can conclude that:
%\begin{equation}
%Q^{\pi_M^*}\left(s,\pi_M^*(s)\right) - Q^{\pi_A^*}\left(s,\pi_M^*(s)\right) \leq g^{-1}(\epsilon)
%\end{equation}
%
%Where $g^{-1}(\epsilon)$ is (1) well defined, (2) at most a polynomial function of epsilon. If the above properties don't get us there, what would?

\newpage
% Subsection: phi_a^* Results
\section{$\phi_{a^*}$ Results}




% --- BIBLIOGRAPHY ---
\newpage
\bibliography{sca_bib}

\end{document}
