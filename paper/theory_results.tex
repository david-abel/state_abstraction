% --- SECTION: State Abstraction ---
\section{Approximate State Abstraction}

% Intro to approx state abstraction results.
Here, we introduce our formal method for analyzing approximate state abstraction, including results bounding the error associated with these abstraction methods. In particular, we demonstrate that abstractions based on $Q^*$ similarity, model similarity (that is, $\mathcal{T}$ and $\mathcal{R}$), and similarity between distributions over $Q^*$, for both multinomial and Boltzmann distributions, induce abstract \acp{MDP} for which the optimal policy has bounded error in the ground MDP.

In general, our proof strategy for each of the included results is as follows:
\begin{itemize}
\item First, we relate the values of states in the ground MDP to the value of states in the abstract \ac{MDP},
\item then, we relate the value of states in the ground \ac{MDP} to the value of states in the abstract \ac{MDP} under the optimal abstract policy.
\end{itemize}

% zzz that's odd... sounds like the same thing twice but with optimal policy added. we'll see how that works.
% In response^^^ Could add a sentence pointing out how these differ.

We now introduce some additional notation to formally discuss these results.

{\bf $\ep_f$ definition:} Given a function $f: \mathcal{S}_G \times \mathcal{A} \rightarrow \mathcal{R}$ and a fixed non-negative $\epsilon \in \mathbb{R}$, we define $\ep_f$ as a type of approximate state-aggregation function that satisfies the following for any two ground states $s_1$ and $s_2$: 
\begin{equation}
\label{eq:phi_f}
\ep_f(s_1) = \ep_f(s_2) \rightarrow \forall_a \left|f(s_1, a) - f(s_2, a)\right| \leq \epsilon
\end{equation}

That is, when $\ep_f$ aggregates states, all aggregated states have values of $f$ within $\epsilon$ of each other for all actions. Note that, for any $f$, there are many functions that satisfy this criterion.

We now introduce the main result of the paper.

% Theorem 1
\begin{thm}
There exist at least four classes of approximate state-aggregation functions, $\ep_{Q^*}$, $\ep_{\text{model}}$, $\ep_{\text{bolt}}$ and $\ep_{\text{mult}}$ for which the optimal policy in the abstract \ac{MDP}, applied to the ground \ac{MDP}, has suboptimality bounded polynomially in $\epsilon$:
\begin{equation}
\forall_{s \in \mathcal{S}_G}: | V^{\pi^*_G}(s) - V^{\pi^*_{A}}(s) | \leq poly(\epsilon).
\end{equation}

\end{thm}

\enote{Consider moving definition of $V^{\pi^*_{A}}(s)$ so it's fresh for reader.}
% zzz Also, I think it might be worth spelling out the poly(eps) part. Seems incomplete this way.

% Subsection: Optimal Q Function
\subsection{Optimal Q Function: $\ep_{Q^*}$}

% Q* Lemma
Recall that $Q^*$ is a function on $\mathcal{S}_G \times \mathcal{A}$ that satisfies \enote{$Q^*$ definition}. We consider an approximate version of the $Q^*$-irrelevance abstraction. An approximate $Q$ function has the same form as Equation~\ref{eq:phi_f}:
\begin{equation}
\ep_{Q^*}(s_1) = \ep_{Q^*}(s_2) \rightarrow \forall_a \left|Q^*(s_1, a) - Q^*(s_2, a)\right| \leq \epsilon.
\end{equation}

\begin{lma}
\label{lma:Q*}
When $\ep_{Q^*}$ is used to create the abstract state space $\mathcal{S}_A$:
\begin{equation}
\forall_{s \in \mathcal{S}_G}: | V(s) - V^{\pi^*_{A}}(s) | \leq \frac{2\epsilon}{(1-\gamma)^2}.
\end{equation}
\end{lma}

\textbf{Proof:}
We begin by showing that Q values in the abstract \ac{MDP} closely resemble those in the ground \ac{MDP}.

Let $Q_G: \mathcal{S}_G \times \mathcal{A} \rightarrow \mathbb{R}$ and $V_G: \mathcal{S}_G \rightarrow \mathbb{R}$ denote the Q and value functions in the ground \ac{MDP}.

Let $Q_A: \mathcal{S}_A \times \mathcal{A} \rightarrow \mathbb{R}$ and $V_A: \mathcal{S}_A \rightarrow \mathbb{R}$  stand for the Q and value functions in the abstract \ac{MDP}.

% zzz I don't understand what ``the'' Q function means. if it's the optimal values, we already had a notation for these and I don't see why there should be another.

Let $X(s) = \{g \in \mathcal{S}_G | \epQ(g) = s\}$, the set of ground states that correspond to abstract state $s$.

\begin{clm}
Q-Values in the abstract \ac{MDP} closely resemble those in the ground \ac{MDP}:
\begin{equation}
\label{eq:Q*Claim1}
\forall_{s_G \in \mathcal{S}_G, a} |Q_G(s_G, a) - Q_A(\epQ(s_G), a)| \leq \frac{\epsilon}{1-\gamma}.
\end{equation}
\end{clm}

Consider a temporally heterogeneous \ac{MDP}, $M^T = \langle \mathcal{S}_T, \mathcal{A}_G, \mathcal{R}_T, \mathcal{T}_T, \gamma \rangle$, parameterized by integer $T$, such that for the first $T$ time steps the reward function, transition dynamics and state space are those of the abstract MDP, $M_A$, and after $T$ time steps the reward function, transition dynamics and state spaces are those of the ground MDP, $M_G$. Thus,
\begin{equation}
\mathcal{S}_T = \begin{cases}
\mathcal{S}_G,& T \leq 0, \\
\mathcal{S}_A,& \text{o/w}, \\
\end{cases}
\end{equation}
\begin{equation}
\mathcal{R}_T(s,a) = \begin{cases}
\mathcal{R}_G(s,a),& T \leq 0, \\
\mathcal{R}_A(s, a),& \text{o/w}, \\
\end{cases}
\end{equation}
and
\begin{equation}
\mathcal{T}_T(s,a,s') = \begin{cases}
\mathcal{T}_G(s,a,s'),& T \leq 0, \\
\underset{{g \in X(s)}}{\sum}\left[w(g)T_G(g, a, s')\right],& T = 1, \\
\mathcal{T}_A(s,a,s'),& \text{o/w}. \\
\end{cases}
\end{equation}

It follows that the Q-value of a state $s$ in $M^T$ for action $a$ is defined by:
% zzz too long... needs to be broken up.
% zzz also, I'm not sure we want the temporally heterogeneous MDP to change state spaces mid stream. I think it should all be in terms of the ground MDP (but with the phi function used as needed)
% zzz also, ``w(g)'' comes out of the blue... there's got to be a better notation here.
\dnote{Let's put a reminder of the weighting scheme here}
\begin{equation}
Q_T(s, a) = 
\begin{cases}
	   Q_G(s, a) &  T=0\\
	   \underset{g \in X(s)}{\sum} \left[ w(g)Q_G(g,a) \right] & T = 1\\
	   \mathcal{R}_A(\epQ(s),a) + \gamma \underset{{s_A}' \in \mathcal{S}_A}{\sum} \mathcal{T}_A(\epQ(s),a,{s_A}') \max_{a'} Q_{T-1}({s_A}', a') & \text{o/w}
\end{cases}
\end{equation}

We proceed by induction on $T$ to show that:
\begin{equation}
\forall_{T, s_G \in \mathcal{S}_G, a} |Q_0(s_G, a) - Q_T(s_G, a)| \leq \sum_{t=0}^{T-1} \epsilon \gamma^{t},
\end{equation}
that is, the Q function remains relatively unchanged over time.

% Base Case t=0
\textit{Base Case: $T = 0$}

When $T = 0$, $Q_T = Q_0$, so the result trivially follows.

% Base Case t=1
\textit{Base Case: $T = 1$}
\begin{align*}
&Q_1(s,a) = \underset{g \in X(s)}{\sum} \left[ w(g)Q_G(g,a) \right].\\
\end{align*}
Since all co-aggregated states have Q-values within $\epsilon$ of one another and $w(g)$ induces a convex combination,
\begin{align*}
&Q_1(s,a) \leq \epsilon \gamma^t + \epsilon + Q_0(s_G, a),\\
&Q_1(s,a) \leq \gamma\sum_{t=0}^{1} \epsilon \gamma^t + Q_0(s_G, a).\\
&\text{Thus,}\\
&Q_1(s,a) \left| Q_{T}(s_A, a) - Q_0(s_G,a) \right| \leq \gamma\sum_{t=0}^{T}\epsilon \gamma^t
\end{align*}



% Inductive Case
\textit{Inductive Case: $T > 1$}

We assume as our inductive hypothesis that:
\begin{equation}
\forall_{s_G \in \mathcal{S}_G, a} |Q_0(s_G, a) - Q_{T-1}(\epQ(s_G), a)| \leq \sum_{t=0}^{T-2} \epsilon \gamma^t.\\
\end{equation}

Consider a fixed but arbitrary state, $s_G \in \mathcal{S}_G$, and fixed but arbitrary action $a$.

We denote $\epQ(s_G)$ as $s_A$.
\begin{align*}
&\text{By definition of $Q_{T}(s_A, a)$, $\mathcal{R}_A$, $\mathcal{T}_A$:}\\
&Q_T(s_A, a) =\sum_{g \in X(s_A)}w(g)\left[ R_0(g,a) + \gamma \sum_{g' \in \mathcal{S}_G} T_0(g,a,g') \max_{a'} Q_{T-1}(g', a')      \right]\\
&\text{Applying our inductive hypothesis yields}\\
&\leq \sum_{g \in X(s_A)}w(g)\left[ R_0(g,a) + \gamma \sum_{g' \in \mathcal{S}_G} T_0(g,a,g') \max_{a'}(Q_0(g', a') + \sum_{t=0}^{T-2} \epsilon \gamma^t)      \right]\\
&\leq \gamma\sum_{t=0}^{T-2} \epsilon \gamma^t + \sum_{g \in X(s_A)}w(g)\left[ R_0(g,a) + \gamma \sum_{g' \in \mathcal{S}_G} T_0(g,a,g') \max_{a'}Q_0      \right]\\
&\leq \gamma\sum_{t=0}^{T-2} \epsilon \gamma^t + \sum_{g \in X(s_A)}\left[ w(g)\ Q_0(g,a)\right]\\
&\text{Since all aggregated states have Q-values within $\epsilon$ of one another:}\\
&\leq \gamma\sum_{t=0}^{T-2} \epsilon \gamma^t + \epsilon + Q_0(s_G, a)\\
&\leq \gamma\sum_{t=0}^{T-1} \epsilon \gamma^t + Q_0(s_G, a)\\
&\text{Thus,}\\
&\left| Q_{T}(s_A, a) - Q_0(s_G,a) \right| \leq \gamma\sum_{t=0}^{T-1}\epsilon \gamma^t
\end{align*}

Since $s_G$ and $s_A$ are arbitrary and $Q_0=Q_G$ we conclude
\begin{equation}
\forall_{T, g \in \mathcal{S}_G, a} |Q_0(g, a) - Q_T(\epQ(g), a)| \leq \sum_{t=0}^{T-1} \epsilon \gamma^t
\end{equation}

As $T \rightarrow \infty$, $Q_T$ becomes $Q_A$ and $\sum_{t=0}^{T-1} \epsilon \gamma^t$ becomes $\frac{\epsilon}{1-\gamma}$. Therefore we conclude Equation $\ref{eq:Q*Claim1}$.

Consider a fixed but arbitrary state, $s_G \in \mathcal{S}_G$ and its corresponding abstract state $s_A=\epQ(s_G)$.

Let $a^*_G$ stand for the optimal action for $s_G$: \\
$a^*_G = \argmin Q_G(s_G, a)$

Let $a^*_A$ stand for the optimal action for $s_A$:\\
 $a^*_A = \argmin Q_A(s_A, a)$

%Claim 2
\begin{clm}
\label{clm:optAbsActionNearOptGround}
The optimal action in the abstract MDP has a Q value in the ground which is nearly optimal.
\end{clm}

\begin{equation}
\label{eq:Q*Claim2}
V_G(s_G) \leq Q_G(s_A, a^*_A) + \frac{2\epsilon}{1-\gamma}
\end{equation}


By Equation \ref{eq:Q*Claim1}
\begin{align}
&V_G(s_G) = Q_G(s_G, a^*_G) \leq Q_A(s_G, a^*_G) + \frac{\epsilon}{1-\gamma} \\
\label{eq:Q*OptActionResult}
\end{align}

 By the definition of $a^*_A$ we know that 
 \begin{align}
Q_A(s_G, a^*_G) + \frac{\epsilon}{1-\gamma} \leq Q_A(s_A, a^*_A) + \frac{\epsilon}{1-\gamma}
\end{align}

Lastly again by Equation \ref{eq:Q*Claim1} we know
\begin{align}
Q_A(s_A, a^*_A) + \frac{\epsilon}{1-\gamma} \leq Q_G(s_A, a^*_A) + \frac{2\epsilon}{1-\gamma}
\end{align}

Therefore, we conclude Equation \ref{eq:Q*Claim2}.

%Claim 3
\begin{clm}
We conclude Lemma \ref{lma:Q*} by Claim \ref{clm:optAbsActionNearOptGround}.
\end{clm}

Consider the policy for $M_G$ of following the optimal abstract policy $\pi^*_A$ for t steps and then following the optimal ground policy $\pi^*_G$ in $M_G$:

\begin{equation}
\pi_{A,t}(s)=
\begin{cases}
\pi_G^*(s), \text{if $t<=0$}\\
\pi_A^*(\ep(s)), \text{if $t > 0$}\\
\end{cases}
\end{equation}

For a particular $t$, the value of this policy for $s_G \in \mathcal{S}_G$ is as follows:
\begin{align*}
V^{\pi_{A,t}}(s_G) &= \\
&R_G(s, \pi_{A,t}(s_G)) + \gamma \sum_{{s_G}' \in \mathcal{S}_G}\mathcal{T}_G(s_G, a, {s_G}')V^{\pi_{A,t-1}}({s_G}')
\end{align*}

%Start induction on following the optimal abstract policy
We now show by induction on $t$ that
\begin{equation}
\forall_{t, s_G \in \mathcal{S}_g} V^{\pi^*}(s_G) \leq  V^{\pi_{A,t}}(s_G) + \sum_{i=0}^{t}\gamma^i \frac{2\epsilon}{1-\gamma}
\end{equation}


\textit{Base case: $t=0$}

By definition when $t=0$, $V^{\pi_{A,t}} = V^{\pi^*}$ so our bound trivially holds in this case.

\textit{Inductive case: $t > 0$}

Consider a fixed but arbitrary state in $\mathcal{S}_G$, $s_G$.

We assume for our inductive hypothesis that
\begin{equation}
V^{\pi^*}(s_G) \leq V^{\pi_{A,t-1}}(s_G)  + \sum_{i=0}^{t-1}\gamma^i \frac{2\epsilon}{1-\gamma}
\end{equation}

By definition 
\begin{align*}
&V^{\pi_{A,t}}(s_G) = R_G(s, \pi_{A,t}(s_G)) + \gamma \sum_{g'}\mathcal{T}_G(s_G, a, {s_G}')V^{\pi_{A,t-1}}({s_G}')\\
&\text{Applying our inductive hypothesis yields}\\
&V^{\pi_{A,t}}(g, a) \geq\\
&R_G(s, \pi_{A,t}(g, a)) + \gamma \sum_{g'}\mathcal{T}_G(g, a, g')\left(V^{\pi^*}({s_G}') - \sum_{i=0}^{t-1}\gamma^i \frac{2\epsilon}{1-\gamma} \right)\\
&\geq -\gamma\sum_{i=0}^{t-1}\gamma^i \frac{2\epsilon}{1-\gamma} + R_G(s, \pi_{A,t}(g)) + \gamma \sum_{g'}\mathcal{T}_G(g, a, g')V^{\pi^*}(g')\\
&\geq -\gamma\sum_{i=0}^{t-1}\gamma^i \frac{2\epsilon}{1-\gamma} + Q^{\pi^*}(g, \pi_{A,t} (g))\\
&\text{Applying Equation \ref{eq:Q*Claim2} yields}\\
&\geq \gamma\sum_{i=0}^{t-1}\gamma^i \frac{2\epsilon}{1-\gamma} - \frac{2\epsilon}{1-\gamma} + V_{G}(s_G)\\
&\leq \sum_{i=0}^{t}\gamma^i \frac{2\epsilon}{1-\gamma} + V_G(g)\\
& \text{Thus,}\\
&V^{\pi^*}(s_G) \leq V^{\pi_{A,t}}(s_G)  + \sum_{i=0}^{t}\gamma^i \frac{2\epsilon}{1-\gamma}
\end{align*}

Since $s_G$ was arbitrary, we conclude that our bound holds for all states in $\mathcal{S}_G$ for the inductive case.

Thus, from our base case and induction we conclude that
\begin{equation}
\forall_{t, g \in \mathcal{S}_g} V^{\pi^*}(g) \leq  V^{\pi_{A,t}}(g) + \sum_{i=0}^{t}\gamma^i \frac{2\epsilon}{1-\gamma}
\end{equation}

Note that as $t \rightarrow \infty$, $\sum_{i=0}^{t}\gamma^i \frac{2\epsilon}{1-\gamma} \rightarrow \frac{2\epsilon}{(1-\gamma)^2}$ by the sum of geometric series and $\pi_{A,t}(s) \rightarrow \pi_A^*(\epQ(s))$.

Thus, we conclude
\begin{equation*}
\forall_{s_G \in \mathcal{S}_g} V^{\pi^*}(s_G) \leq  V^{\pi_{A}^*}(s_G) + \frac{2\epsilon}{(1-\gamma)^2}
\end{equation*}

\begin{equation*}
\forall_{s_G \in \mathcal{S}_g} \left | V^{\pi^*}(s_G) - V^{\pi_{A}^*}(s_G) \right | \leq  \frac{2\epsilon}{(1-\gamma)^2}
\end{equation*}



% Subsection: Model Similarity
\subsection{Model Similarity: $\ep_{model}$}

Inspired by the model-irrelevance abstraction, we consider the approximate case. We let $\phi_{model}$ define a type of abstraction that, for fixed $\epsilon$, satisfies:
\begin{multline}
\epM(s_1) = \epM(s_2) \rightarrow \\
\forall_a \left| \mcR(s_1, a) - \mcR(s_2, a)\right| \leq \epsilon \wedge\ \\
\forall_{s_A \in \mcS_A} \left|\sum_{g' \in X(s_A)}\mcT(s_1, a, g') - \mcT(s_2, a,g')\right| \leq \epsilon
\end{multline}
where $X(s_A) = \{ g' \in \mcS | \epM(g') = s_A \}$

\enote{Do we want to make it clear that this is ground transitions and rewards?}

\enote{Note that it is possible }

%Model Lemma
\begin{lma}
\label{lma:model}
When $S_A$ is created using a function of the $\ep_{model}$ type:
\begin{equation}
\forall_{s \in S_M} : | V^{\pi^*_G}(s) - V^{\pi^*_{A}}(s) | \leq \frac{2\epsilon + 2\gamma((|\mcS_G|-1)\epsilon)}{(1-\gamma)^3}
\end{equation}
\end{lma}

{\bf Proof:} Our proof proceeds as follows: we demonstrate that states grouped aggregated under $\epM$ have Q values all within a particular bound. Consequently $\epM$ abstractions are interpretable as an instantiation of $\epQ$ with a particular $\epsilon$ -- understanding $\epM$ abstractions allows us to conclude our bound by applying Lemma \ref{lma:Q*}.


Let $B$ stand for the maximum Q difference between any pair of ground states in the same abstract state under $\epM$:

\begin{align*}
B = &\max_{s_A \in \mcS_A, s_1, s_2 \in X(s_A), a} |Q(s_1, a) - Q(s_2, a)|\\
=&\max_{s_A \in \mcS_A, s_1, s_2 \in X(s_A), a}      |R(s_1, a) - R(s_2, a) +\\
& \gamma \sum_{g' \in \mcS_G}(\mcT(s_1,a,g')-\mcT(s_2, a, g'))\max_{a'}Q(g', a')|\\
\leq &\epsilon + \gamma \sum_{s_A \in \mcS_A}\sum_{g' \in X(s_A)}\left[(T(s_1, a, g')-T(s_2, a, g')) 		\max_{a'}Q(g', a')	\right]\\
&\text{By the similarity of transitions of grouped states under $\epM$:}\\
 \leq &\epsilon + \gamma Q_{max} \sum_{s_A \in \mcS_A} \epsilon \\
\leq& \epsilon + \gamma|\mcS_G|\epsilon Q_{max}\\
&\text{Applying 0-1 reward normalization and algebra:}\\
 \leq & \frac{\epsilon + \gamma(|\mcS_G| - 1) \epsilon}{1-\gamma}
\end{align*}

Since the Q values of ground states grouped under $\epM$ are strictly less than $B$, we can understand $\epM$ as a type of $\epQ$ with $\epsilon = B$. Applying Lemma \ref{lma:Q*} yields Lemma \ref{lma:model}.

% Subsection: Multinomial over Optimal Q
\enote{If not here, we should have some motivation for multinomial (and boltz)}
\subsection{Multinomial over Optimal Q: $\ep_{mult}$}

We let $\phi_{mult}$ define a type of abstraction that, for fixed $\epsilon$ satisfies:
\begin{multline}
\ep_{mult}(s_1) = \ep_{mult}(s_2) \rightarrow \\
\forall_{a} \left|\frac{Q^*(s_1,a)}{\sum_b Q^*(s_1,b)} - \frac{Q^*(s_1,a)}{\sum_b Q^*(s_1,b)}\right| \leq \epsilon
\end{multline}

\dnote{We also need to add the point about the denominator bound...}

{\bf Lemma 3:} When $S_A$ is created using a function of the $\ep_{mult}$ type:
\begin{equation}
\forall_{s \in S_M} : | V^{\pi^*_G}(s) - V^{\pi^*_{A}}(s) | \leq \frac{\left(\textsc{QMax}|\mathcal{A}| + k \epsilon + k \epsilon^2\right)}{1-\gamma}:
\end{equation}

{\bf Proof Sketch:}

\dnote{Proof sketch}


% Subsection: Boltzmann over Optimal Q
\subsection{Boltzmann over Optimal Q: $\ep_{bolt}$}

Similarly to the multinomial abstraction, we let $\phi_{bolt}$ define a type of abstraction that, for fixed $\epsilon$ satisfies:
\begin{multline}
\ep_{bolt}(s_1) = \ep_{bolt}(s_2) \rightarrow \\
\forall_{a} \left|\frac{e^{Q^*(s_1,a)}}{\sum_b e^{Q^*(s_1,b)}} - \frac{e^{Q^*(s_1,a)}}{\sum_b e^{Q^*(s_1,b)}}\right| \leq \epsilon
\end{multline}

{\bf Lemma 4:} When $S_A$ is created using a function of the $\ep_{bolt}$ type:
\begin{equation}
\forall_{s \in S_M} : | V^{\pi^*_G}(s) - V^{\pi^*_{A}}(s) | \leq \frac{\epsilon(\textsc{QMax}|\mathcal{A}| + k\epsilon) + A}{1-\gamma}:
\end{equation}

{\bf Proof Sketch:}

\dnote{Proof sketch}


\dnote{(Probably put full proofs in appendix if we have room?}

% There are at least four classes of functions:
% 1 \phi_q^*
% 2 \phi_model
% 3 \phi_mult
% 4 \phi_bolt


% Explain each of the Phis

% One theorem, each separate phi is its own lemma.

% Proof of Q^*

% Possibly a lemma for NP-Completeness of the minimal such phi for all abstractions (was a result for model in previous cases).



% Subsection: Other Abstractions
\subsection{Other Abstractions}

We note that one natural way of approximating $\phi_{a^*}$ from ~\cite{li2006towards}, in which states that are compressed together share optimal actions and the Q values of these actions are within $\varepsilon$ is ultimately equivalent to the crisp abstraction $\phi_{\pi^*}$, in states that are compressed share an optimal action. A true approximation of $\phi_{a^*}$ ought to also approximate the optimality of each action. Given the degree of compression achievable under $\phi_{a^*}$, especially with temporally extended actions, we foresee this approximate form of abstraction as being of great interest, and plan to investigate it in future work.

We are {\it not} going to provide results for $\phi_{\pi^*}$, since relaxing equality of optimal actions doesn't mean anything.

Additionally, we are {\it not} going to provide results for $\phi_{Q^\pi}$, since as Lihong's paper notes, ``It is an open question how to find $Q^\pi$ irrelevance abstractions without enumerating all possible policies, they do not give results. Furthermore, an MDP for which this is true is an awfully weird MDP...

Lastly, we are interested in a generalization of $\phi_{mult}$  and $\phi_{bolt}$ that handles a broader space of distributions over Q values.

We also note that any abstraction that depends only on the reward function, $\mathcal{R}$, can incur unbounded error (or rather, $\textsc{VMax}$).`

% Some abstractions DON'T preserve a meaningful notion of optimality.


