% --- SECTION: State Abstraction ---
\section{Approximate State Abstraction}

% Intro to approx state abstraction results.
Here, we introduce our formal method for analyzing approximate state abstraction, including results bounding the error associated with these abstraction methods. In particular, we demonstrate that abstractions based on approximate $Q^*$ similarity (\ref{sec:Q*}), approximate model similarity (\ref{sec:model}), and approximate similarity between distributions over $Q^*$, for both Boltzmann (\ref{sec:boltz}) and multinomial (\ref{sec:mult}) distributions, induce abstract \acp{MDP} for which the optimal policy has bounded error in the ground MDP.

%In general, our proof strategy for each of the included results is as follows:
%\begin{itemize}
%\item First, we relate the values of states in the ground MDP to the value of states in the abstract \ac{MDP},
%\item then, we relate the value of states in the ground \ac{MDP} to the value of states in the abstract \ac{MDP} under the optimal abstract policy.
%\end{itemize}

% Explanation of the above.
%The first statement is relating Q values defined over the ground MDP with Q values defined over the abstract MDP, while the second statement is comparing the optimal Q values in the ground MDP, with the Q values in the ground MDP under the abstract policy. This is because our learners will solve for the optimal policy in the abstract, and use it in the ground MDP by mapping each ground state to its abstract state and querying the abstract policy for behavior. That is:
%\begin{equation}
%\pi_G(s) = \pi_A(\phi(s))
%\end{equation}


% zzz that's odd... sounds like the same thing twice but with optimal policy added. we'll see how that works.
% In response^^^ Could add a sentence pointing out how these differ.

We now introduce some additional notation to formally discuss these results.

\bdefn{$\pi_A^*$, $\pi_G^*$}
We let $\pi_A^* : \mcS_A \rightarrow \mcA$ and $\pi_G^* : \mcS_G \rightarrow \mcA$ stand for the optimal policies in the abstract and ground \acp{MDP} respectively.
\edefn

% Abstract policy in ground.
We are interested in how the optimal policy in the abstract \ac{MDP} performs in the ground \ac{MDP}. As such we formally define the policy in the ground \ac{MDP} derived from optimal behavior in the abstract \ac{MDP}:

\bdefn{$\pi_{GA}$}
Given a $s \in \mcS_G$ and a state aggregation function, $\phi$, 
\begin{equation}
\pi_{GA}(s)=\pi_A^*(\phi(s))
\end{equation}
\edefn

We now define classes of abstraction based on functions state, action pairs.
\bdefn{$\ep_f$}
Given a function $f: \mathcal{S}_G \times \mathcal{A} \rightarrow \mathcal{R}$ and a fixed non-negative $\epsilon \in \mathbb{R}$, we define $\ep_f$ as a type of approximate state-aggregation function that satisfies the following for any two ground states $s_1$ and $s_2$: 
\begin{equation}
\label{eq:phi_f}
\ep_f(s_1) = \ep_f(s_2) \rightarrow \forall_a \left|f(s_1, a) - f(s_2, a)\right| \leq \epsilon
\end{equation}
\edefn

That is, when $\ep_f$ aggregates states, all aggregated states have values of $f$ within $\epsilon$ of each other for all actions.

\bdefn{$Q_G$, $V_G$}
Let $Q_G = Q^{\pi_G^*} : \mathcal{S}_G \times \mathcal{A} \rightarrow \mathbb{R} $ and $V_G = V^{\pi_G^*}: \mathcal{S}_G \rightarrow \mathbb{R} $ denote the optimal Q and optimal value functions in the ground \ac{MDP}.
\edefn

\bdefn{$Q_A$, $V_A$}
Let $Q_A  = Q^{\pi_A^*}: \mathcal{S}_A \times \mathcal{A} \rightarrow \mathbb{R}$ and $V_A  = V^{\pi_A^*}: \mathcal{S}_A \rightarrow \mathbb{R}$  stand for the optimal Q and optimal value functions in the abstract \ac{MDP}.
\edefn

We now introduce the main result of the paper.

% Theorem 1
\begin{thm}
There exist at least four classes of approximate state-aggregation functions, $\ep_{Q^*}$, $\ep_{\text{model}}$, $\ep_{\text{bolt}}$ and $\ep_{\text{mult}}$ for which the optimal policy in the abstract \ac{MDP}, applied to the ground \ac{MDP}, has suboptimality bounded polynomially in $\epsilon$:
\begin{equation}
\forall_{s \in \mathcal{S}_G}: | V_G^{\pi_{GA}}(s) - V_G^{\pi_G^*}(s) | \leq poly(\epsilon).
\end{equation}
\end{thm}
\enote{Axe the absolute values here and elsewhere}

\enote{Unify the use of $\forall$ and ``:"s}

\enote{Add for 0-1 normalized reward}

\enote{Add note relating this all to Lihong for eps=0 and cut down the Lihong section}

% Subsection: Optimal Q Function
\subsection{Optimal Q Function: $\ep_{Q^*}$}
\label{sec:Q*}

% Q* Lemma -- lemma 1
We consider an approximate version of the $Q^*$-irrelevance abstraction. 

\bdefn{$\epQ$}
An approximate $Q$ function has the same form as Equation~\ref{eq:phi_f}:
\begin{equation}
\ep_{Q^*}(s_1) = \ep_{Q^*}(s_2) \rightarrow \forall_a \left|Q_G^*(s_1, a) - Q_G^*(s_2, a)\right| \leq \epsilon.
\end{equation}
\edefn

% Q^* Lemma
\begin{lma}
\label{lma:Q*}
When $\ep_{Q^*}$ is used to create the abstract state space $\mcS_A$:
\begin{equation}
\forall_{s \in \mcS_G}: | V_G^{\pi_G^*}(s) - V_G^{\pi_{GA}}(s) | \leq \frac{2\epsilon}{(1-\gamma)^2}.
\end{equation}
\end{lma}

% Proof
\textbf{Proof:}
\enote{I would like a quick overview of the proof here if nobody objects.}
% Claim 1:
\begin{clm}
\label{clm:closeQs}
Optimal $Q$ values in the abstract \ac{MDP} closely resemble optimal $Q$ values in the ground \ac{MDP}:
\begin{equation}
\label{eq:Q*Claim1}
\forall_{s_G \in \mathcal{S}_G, a} |Q_G(s_G, a) - Q_A(\epQ(s_G), a)| \leq \frac{\epsilon}{1-\gamma}.
\end{equation}
\end{clm}

Consider a temporally heterogeneous \ac{MDP}, $M^T = \langle \mathcal{S}_T, \mathcal{A}_G, \mathcal{R}_T, \mathcal{T}_T, \gamma \rangle$, parameterized by integer $T$, such that for the first $T$ time steps the reward function, transition dynamics and state space are those of the abstract MDP, $M_A$, and after $T$ time steps the reward function, transition dynamics and state spaces are those of the ground MDP, $M_G$. Thus,
\begin{equation}
\mathcal{S}_T = \begin{cases}
\mathcal{S}_G,& T \leq 0, \\
\mathcal{S}_A,& \text{o/w},
\end{cases}
\end{equation}
\begin{equation}
\mathcal{R}_T(s,a) = \begin{cases}
\mathcal{R}_G(s,a),& T \leq 0, \\
\mathcal{R}_A(s, a),& \text{o/w},
\end{cases}
\end{equation}
\begin{equation}
\mcT_T(s,a,s') = \begin{cases}
\mcT_G(s,a,s'),& T \leq 0, \\
\underset{{g \in G(s)}}{\sum}\left[w(g)\mcT_G(g, a, s')\right],& T = 1, \\
\mcT_A(s,a,s'),& \text{o/w}.
\end{cases}
\end{equation}

It follows that the $Q$-value of a state $s$ in $M^T$ for action $a$ is defined by:
% zzz too long... needs to be broken up.
\begin{equation}
Q_T(s, a) = 
\begin{cases}
	   Q_G(s, a) &  T=0\\
	   \underset{g \in G(s)}{\sum} \left[ w(g)Q_G(g,a) \right] & T = 1\\
	   \mathcal{R}_A(\epQ(s),a) + \sigma_{T-1}(s,a) &\text{o/w}

\end{cases}
\end{equation}

Where:
\begin{equation}
\sigma_{T-1}(s,a) = \gamma \underset{{s_A}' \in \mathcal{S}_A}{\sum} \mathcal{T}_A(\epQ(s),a,{s_A}') \max_{a'} Q_{T-1}({s_A}', a')
\end{equation}

We proceed by induction on $T$ to show that:
\begin{equation}
\forall_{T, s_G \in \mathcal{S}_G, a} |Q_G(s_G, a) - Q_T(s_G, a)| \leq \sum_{t=0}^{T-1} \epsilon \gamma^{t},
\end{equation}

% Base Case t=0
\textit{Base Case: $T = 0$}

When $T = 0$, $Q_T = Q_G$, so the result trivially follows.

% Base Case t=1
\textit{Base Case: $T = 1$}
\begin{align*}
&Q_1(s,a) = \underset{g \in G(s)}{\sum} \left[ w(g)Q_G(g,a) \right].
\end{align*}
Since all co-aggregated states have Q-values within $\epsilon$ of one another and $w(g)$ induces a convex combination,
\begin{align*}
&Q_1(s,a) \leq \epsilon \gamma^t + \epsilon + Q_G(s_G, a),
\end{align*}
Thus,
\begin{equation}
\left| Q_{1}(s_A, a) - Q_G(s_G,a) \right| \leq \sum_{t=0}^{1}\epsilon \gamma^t
\end{equation}
% Inductive Case
\textit{Inductive Case: $T > 1$}

We assume as our inductive hypothesis that:
\begin{equation}
\forall_{s_G \in \mathcal{S}_G, a} |Q_G(s_G, a) - Q_{T-1}(\epQ(s_G), a)| \leq \sum_{t=0}^{T-2} \epsilon \gamma^t.
\end{equation}

Consider a fixed but arbitrary state, $s_G \in \mathcal{S}_G$, and fixed but arbitrary action $a$.

We denote $\epQ(s_G)$ as $s_A$. 

By definition of $Q_{T}(s_A, a)$, $\mathcal{R}_A$, $\mathcal{T}_A$:
\begin{multline*}
Q_T(s_A, a) = \sum_{g \in G(s_A)}w(g)\ * \\ 
 \left[ R_0(g,a) + \gamma \sum_{g' \in \mathcal{S}_G} T_0(g,a,g') \max_{a'} Q_{T-1}(g', a')      \right]
\end{multline*}
Applying our inductive hypothesis yields,
\begin{multline*}
\leq \sum_{g \in G(s_A)}w(g)\ * \\ \left[ R_G(g,a) + \gamma \sum_{g' \in \mathcal{S}_G} T_G(g,a,g') \max_{a'}(Q_G(g', a') + \sum_{t=0}^{T-2} \epsilon \gamma^t)      \right]
\end{multline*}
%&\leq \gamma\sum_{t=0}^{T-2} \epsilon \gamma^t + \sum_{g \in X(s_A)}w(g)\left[ R_0(g,a) + \gamma \sum_{g' \in \mathcal{S}_G} T_0(g,a,g') \max_{a'}Q_0      \right]\\
Then:
\begin{equation*}
\leq \gamma\sum_{t=0}^{T-2} \epsilon \gamma^t + \sum_{g \in G(s_A)}\left[ w(g)\ Q_G(g,a)\right]
\end{equation*}
Since all aggregated states have Q-values within $\epsilon$ of one another:
\begin{align*}
\leq \gamma\sum_{t=0}^{T-2} \epsilon \gamma^t + \epsilon + Q_G(s_G, a) \\
\therefore \left| Q_{T}(s_A, a) - Q_G(s_G,a) \right| &\leq \gamma\sum_{t=0}^{T-1}\epsilon \gamma^t
\end{align*}
Since $s_G$ and $s_A$ are arbitrary we conclude
\begin{equation}
\forall_{T, s_G \in \mathcal{S}_G, a} |Q_G(s_G, a) - Q_T(\epQ(s_G), a)| \leq \sum_{t=0}^{T-1} \epsilon \gamma^t
\end{equation}

As $T \rightarrow \infty$, $Q_T$ becomes $Q_A$ and $\sum_{t=0}^{T-1} \epsilon \gamma^t$ becomes $\frac{\epsilon}{1-\gamma}$. Therefore we conclude Equation $\ref{eq:Q*Claim1}$. \\

%Claim 2
\begin{clm}
\label{clm:optAbsActionNearOptGround}
%The optimal action in the abstract MDP has a $Q$-value in the ground which is nearly optimal:

Consider a fixed but arbitrary state, $s_G \in \mathcal{S}_G$ and its corresponding abstract state $s_A=\epQ(s_G)$.

Let $a^*_G$ stand for the optimal action in $s_G$:
\[a^*_G = \argmin Q_G(s_G, a)\]

Let $a^*_A$ stand for the optimal action in $s_A$:
 \[a^*_A = \argmin Q_A(s_A, a)\]

The optimal action in the abstract MDP has a $Q$-value in the ground which is nearly optimal:
\begin{equation}
\label{eq:Q*Claim2}
V_G(s_G) \leq Q_G(s_A, a^*_A) + \frac{2\epsilon}{1-\gamma}
\end{equation}
\end{clm}

By Claim \ref{clm:closeQs}:
\begin{align}
&V_G(s_G) = Q_G(s_G, a^*_G) \leq Q_A(s_G, a^*_G) + \frac{\epsilon}{1-\gamma}
\label{eq:Q*OptActionResult}
\end{align}

By the definition of $a^*_A$ we know that 
\begin{align}
Q_A(s_G, a^*_G) + \frac{\epsilon}{1-\gamma} \leq Q_A(s_A, a^*_A) + \frac{\epsilon}{1-\gamma}
\end{align}

Lastly again by Claim \ref{clm:closeQs} we know
\begin{align}
Q_A(s_A, a^*_A) + \frac{\epsilon}{1-\gamma} \leq Q_G(s_A, a^*_A) + \frac{2\epsilon}{1-\gamma}
\end{align}

Therefore, we conclude Equation \ref{eq:Q*Claim2}.

%Claim 3
\begin{clm}
Lemma \ref{lma:Q*} follows from Claim \ref{clm:optAbsActionNearOptGround}.
\end{clm}

Consider the policy for $M_G$ of following the optimal abstract policy $\pi^*_A$ for t steps and then following the optimal ground policy $\pi^*_G$ in $M_G$:
\begin{equation}
\pi_{A,t}(s)=
\begin{cases}
\pi_G^*(s), \text{if $t<=0$}\\
\pi_{GA}(s), \text{if $t > 0$}
\end{cases}
\end{equation}

For a particular $t$, the value of this policy for $s_G \in \mathcal{S}_G$ in the ground \ac{MDP} is as follows:
\begin{multline*}
V_G^{\pi_{A,t}}(s_G) = \\
R_G(s, \pi_{A,t}(s_G)) +\ \gamma \sum_{{s_G}' \in \mathcal{S}_G}\mathcal{T}_G(s_G, a, {s_G}')V_G^{\pi_{A,t-1}}({s_G}')
\end{multline*}
%Start induction on following the optimal abstract policy
We now show by induction on $t$ that
\begin{equation}
\forall_{t, s_G \in \mathcal{S}_g} V_G(s_G) \leq  V_G^{\pi_{A,t}}(s_G) + \sum_{i=0}^{t}\gamma^i \frac{2\epsilon}{1-\gamma}
\end{equation}

\textit{Base case: $t=0$}

By definition when $t=0$, $V_G^{\pi_{A,t}} = V_G$, so our bound trivially holds in this case.

\textit{Inductive case: $t > 0$}

Consider a fixed but arbitrary state in $\mathcal{S}_G$, $s_G$.

We assume for our inductive hypothesis that
\begin{equation}
V_G(s_G) \leq V_G^{\pi_{A,t-1}}(s_G) + \sum_{i=0}^{t-1}\gamma^i \frac{2\epsilon}{1-\gamma}
\end{equation}
By definition 
\begin{multline*}
V_G^{\pi_{A,t}}(s_G) = R_G(s, \pi_{A,t}(s_G)) + \\ \gamma \sum_{g'}\mathcal{T}_G(s_G, a, {s_G}')V_G^{\pi_{A,t-1}}({s_G}')
\end{multline*}
Applying our inductive hypothesis yields:
\begin{multline*}
V_G^{\pi_{A,t}}(s_G, a) \geq R_G(s_G, \pi_{A,t}(s_G, a))\ + \\ \gamma \sum_{{s_G}'}\mathcal{T}_G(s_G, a, {s_G}')\left(V_G({s_G}') - \sum_{i=0}^{t-1}\gamma^i \frac{2\epsilon}{1-\gamma} \right)
\end{multline*}
Therefore:
\begin{align*}
%&\geq -\gamma\sum_{i=0}^{t-1}\gamma^i \frac{2\epsilon}{1-\gamma} + R_G(s, \pi_{A,t}(g)) + \gamma \sum_{g'}\mathcal{T}_G(g, a, g')V^{\pi^*}(g')\\
&\geq -\gamma\sum_{i=0}^{t-1}\gamma^i \frac{2\epsilon}{1-\gamma} + Q_G(s_G, \pi_{A,t} (s_G))
\end{align*}
Applying Claim \ref{clm:optAbsActionNearOptGround} yields:
\begin{align*}
\geq -\gamma\sum_{i=0}^{t-1}\gamma^i \frac{2\epsilon}{1-\gamma} - \frac{2\epsilon}{1-\gamma} + V_{G}(s_G) \\
%&\leq \sum_{i=0}^{t}\gamma^i \frac{2\epsilon}{1-\gamma} + V_G(s_G)
\therefore V_G(s_G) \leq V_G^{\pi_{A,t}}(s_G)  + \sum_{i=0}^{t}\gamma^i \frac{2\epsilon}{1-\gamma}
\end{align*}
Since $s_G$ was arbitrary, we conclude that our bound holds for all states in $\mathcal{S}_G$ for the inductive case.

Thus, from our base case and induction we conclude that
\begin{equation}
\forall_{t, s_G \in \mathcal{S}_g} V_G^{\pi^*}(s_G) \leq  V_G^{\pi_{A,t}}(s_G) + \sum_{i=0}^{t}\gamma^i \frac{2\epsilon}{1-\gamma}
\end{equation}

Note that as $t \rightarrow \infty$, $\sum_{i=0}^{t}\gamma^i \frac{2\epsilon}{1-\gamma} \rightarrow \frac{2\epsilon}{(1-\gamma)^2}$ by the sum of geometric series and $\pi_{A,t}(s) \rightarrow \pi_{GA}$.

Thus, we conclude Lemma~\ref{lma:Q*}
%\begin{equation*}
%\forall_{s_G \in \mathcal{S}_g} V_G(s_G) \leq  V_G^{\pi_{GA}}(s_G) + \frac{2\epsilon}{(1-\gamma)^2}
%\end{equation*}
%\begin{equation*}
%\forall_{s_G \in \mathcal{S}_g} \left | V_G(s_G) - V_G^{\pi_{GA}}(s_G) \right | \leq  \frac{2\epsilon}{(1-\gamma)^2}
%\end{equation*}

% Subsection: Model Similarity -- lemma 2
\subsection{Model Similarity: $\ep_{model}$}
\label{sec:model}

Inspired by the model-irrelevance abstraction, we consider the approximate case. 
\bdefn{$\epM$}
We let $\epM$ define a type of abstraction that, for fixed $\epsilon$, satisfies:
\begin{multline}
\epM(s_1) = \epM(s_2) \rightarrow \\
\forall_a \left| \mcR_G(s_1, a) - \mcR_G(s_2, a)\right| \leq \epsilon \wedge\ \\
\forall_{s_A \in \mcS_A} \left|\sum_{{s_G}' \in G(s_A)} \left[\mcT_G(s_1, a, {s_G}') - \mcT_G(s_2, a,{s_G}')\right] \right| \leq \epsilon
\end{multline}
\edefn

%Model Lemma
\begin{lma}
\label{lma:model}
When $\mcS_A$ is created using a function of the $\ep_{model}$ type:
\begin{equation}
\forall_{s \in \mcS_G} : | V_G^{\pi_G^*}(s) - V_G^{\pi_{GA}}(s) | \leq \frac{2\epsilon + 2\gamma((|\mcS_G|-1)\epsilon)}{(1-\gamma)^3}
\end{equation}
\end{lma}

{\bf Proof:} Our proof proceeds as follows: we demonstrate that states aggregated under $\epM$ have $Q$-values all within a particular bound. Consequently $\epM$ abstractions are interpretable as an instantiation of $\epQ$ with a particular $\epsilon$. Understanding $\epM$ as such abstractions allows us to conclude our bound by applying Lemma \ref{lma:Q*}.

Let $B$ stand for the maximum $Q$-value difference between any pair of ground states in the same abstract state under $\epM$:
\begin{equation*}
B = \max_{s_A, s_1, s_2, a}  |Q_G(s_1, a) - Q_G(s_2, a)|
\end{equation*}
Where $s_A \in \mcS_G$ and $s_1, s_2 \in G(s_A)$.:
\begin{multline*}
=\max_{s_A, s_1, s_2, a}      |\mcR_G(s_1, a) - \mcR_G(s_2, a) +\\
\gamma \sum_{{s_G}' \in \mcS_G}(\mcT_G(s_1,a,{s_G}')-\mcT_G(s_2, a, {s_G}'))\max_{a'}Q_G({s_G}', a')|
\end{multline*}
\begin{multline*}
\leq \epsilon + \gamma \sum_{s_A \in \mcS_A}\sum_{{s_G}' \in G(s_A)} \\
\left[(T_G(s_1, a, {s_G}')-T_G(s_2, a, {s_G}')) \max_{a'}Q_G({s_G}', a') \right]
\end{multline*}
By the similarity of transitions of grouped states under $\epM$:
\begin{align*}
 \leq &\epsilon + \gamma Q_{max} \sum_{s_A \in \mcS_A} \epsilon \\
\leq& \epsilon + \gamma|\mcS_G|\epsilon Q_{max}
\end{align*}
Applying 0-1 reward normalization and algebra:
\begin{equation*}
 \leq \frac{\epsilon + \gamma(|\mcS_G| - 1) \epsilon}{1-\gamma}
\end{equation*}

\enote{Do we need to state what Q max is and that it is upper bounded by 1 over 1 minus gamma or is this common knowledge?}

Since the $Q$-values of ground states grouped under $\epM$ are strictly less than $B$, we can understand $\epM$ as a type of $\epQ$ with $\epsilon = B$. Applying Lemma \ref{lma:Q*} yields Lemma \ref{lma:model}.

\subsection{Boltzmann over Optimal Q: $\ep_{bolt}$}
\label{sec:boltz}

Here we introduce the approximate abstraction family, $\phi_{bolt}$, which aggregates states with similar Boltzmann distributions on optimal $Q$-values as equivalent. This family of abstractions is appealing as the Boltzmman distribution over $Q$-values has been of interest in the reinforcement learning literature due to its natural balancing of exploration and exploitation~\cite{sutton1998reinforcement}.
\bdefn{$\phi_{bolt}$}
We let $\phi_{bolt}$ define a family of abstractions that, for fixed $\epsilon$ satisfies:
\begin{multline}
\ep_{bolt}(s_1) = \ep_{bolt}(s_2) \rightarrow \\
\forall_{a} \left|\frac{e^{Q_G^*(s_1,a)}}{\sum_b e^{Q_G^*(s_1,b)}} - \frac{e^{Q_G^*(s_2,a)}}{\sum_b e^{Q_G^*(s_2,b)}}\right| \leq \epsilon
\label{eq:phi_bolt}
\end{multline}
\edefn

We also assume that the difference in normalizing terms is bounded by some non-negative constant, $k\in \mathbb{R}$, of $\epsilon$:
\begin{equation}
\left| \sum_b e^{Q_G^*(s_1,b)} - \sum_b e^{Q_G^*(s_2,b)} \right| \leq k*\epsilon
\label{eq:bolt_denom}
\end{equation}
\begin{lma} When $S_A$ is created using a function of the $\ep_{bolt}$ type, for some non-negative constant $k \in \mathbb{R}$::
\begin{equation}
\forall_{s \in \mcS_G} : \left| V_G^{\pi^*_G}(s) - V_G^{\pi_{GA}}(s) \right| \leq \frac{2\epsilon\left(\frac{|\mathcal{A}|}{1-\gamma} + k\epsilon + k\right)}{(1-\gamma)^2}
\end{equation}
\label{lma:bolt_lemma}
\end{lma}

We prove Lemma~\ref{lma:bolt_lemma} by using the approximation for $e^x$:
\begin{equation}
 e^x = 1 + x + \delta  \approx 1 + x
\label{eq:e_to_x_approx}
\end{equation}
Where $\delta$ is the error associated with the approximation.

{\bf Proof Sketch:}

By the approximation in Equation~\ref{eq:e_to_x_approx} and the assumption in Equation~\ref{eq:bolt_denom}:
\begin{align*}
|\frac{1 + Q_G(s_1,a) + \delta_1}{\sum_j e^{Q_G(s_1,a_j)}} - \frac{1 + Q_G(s_2,a) + \delta_2}{\sum_j e^{Q_G(s_1,a_j)} \pm k\epsilon}| \leq \epsilon
\end{align*}

It follows that:
\begin{equation}
|Q_G(s_1,a) - Q_G(s_2,a)| \leq \epsilon \left(\frac{|\mathcal{A}|}{1-\gamma} + k\epsilon + k \right)
\label{eq:bolt_qs}
\end{equation}

Consequently, we can consider $\widetilde{\phi}_{bolt}$ as a type of $\epQ$, where the difference in Q values is bounded by: $\epsilon \left(\frac{|\mathcal{A}|}{1-\gamma} + k\epsilon + k \right)$. Lemma \ref{lma:bolt_lemma} follows directly from Lemma \ref{lma:Q*}.

% Subsection: Multinomial over Optimal Q
\subsection{Multinomial over Optimal Q: $\ep_{mult}$}
\label{sec:mult}

We consider approximate abstractions derived from a multinomial distribution over $Q^*$ for similar reasons to our motivation in considering the Boltzmann distribution.
\bdefn{$\ep_{mult}$}
We let $\phi_{mult}$ define a type of abstraction that, for fixed $\epsilon$ satisfies.
\begin{multline}
\ep_{mult}(s_1) = \ep_{mult}(s_2) \rightarrow \\
\forall_{a} \left|\frac{Q_G^*(s_1,a)}{\sum_b Q_G^*(s_1,b)} - \frac{Q_G^*(s_1,a)}{\sum_b Q_G^*(s_1,b)}\right| \leq \epsilon
\end{multline}
\edefn

We also assume that the difference in normalizing terms is bounded by some positive factor, $k \in \mathbb{R}$, of $\epsilon$:
\begin{equation}
\left |\sum_i Q_G(s_1,a_i) - \sum_j Q_G(s_2,a_j) \right | \leq k\epsilon
\end{equation}
\begin{lma} When $S_A$ is created using a function of the $\ep_{mult}$ type, for some non-negative constant $k \in \mathbb{R}$:
\begin{equation}
\forall_{s \in S_M} : | V_G^{\pi^*_G}(s) - V_G^{\pi_{GA}}(s) | \leq \frac{\frac{2\epsilon|\mathcal{A}|}{1-\gamma} + k \epsilon^2 + k}{(1-\gamma)^2}
\end{equation}
\end{lma}

{\bf Proof Sketch:} The proof follows an identical strategy to that of Lemma~\ref{lma:bolt_lemma} but without the use of the approximation for $e^x$.

\dnote{Bounds for multi and bolt suspiciously similar. Double check math}


% Subsection: Other Abstractions
%\subsection{Other Abstractions}
%
%We note that one natural way of approximating $\phi_{a^*}$ from , in which states that are compressed together share optimal actions and the Q values of these actions are within $\varepsilon$ is ultimately equivalent to the crisp abstraction $\phi_{\pi^*}$, in states that are compressed share an optimal action. A true approximation of $\phi_{a^*}$ ought to also approximate the optimality of each action. Given the degree of compression achievable under $\phi_{a^*}$, especially with temporally extended actions, we foresee this approximate form of abstraction as being of great interest, and plan to investigate it in future work.
%
%We are {\it not} going to provide results for $\phi_{\pi^*}$, since relaxing equality of optimal actions doesn't mean anything.
%
%Additionally, we are {\it not} going to provide results for $\phi_{Q^\pi}$, since as Lihong's paper notes, ``It is an open question how to find $Q^\pi$ irrelevance abstractions without enumerating all possible policies, they do not give results. Furthermore, an MDP for which this is true is an awfully weird MDP...
%
%Lastly, we are interested in a generalization of $\phi_{mult}$  and $\phi_{bolt}$ that handles a broader space of distributions over Q values.
%
%We also note that any abstraction that depends only on the reward function, $\mathcal{R}$, can incur unbounded error (or rather, $\textsc{VMax}$).`

% Some abstractions DON'T preserve a meaningful notion of optimality.


