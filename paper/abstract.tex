The combinatorial explosion that plagues planning and \ac{RL} algorithms can be moderated using abstraction. For instance, prohibitively difficult task representations can be condensed in such a way that essential information is preserved, and consequently, solutions are tractably computable. Because of the difficulty of acquiring perfect knowledge, agents must often perform approximate abstraction on the basis of incomplete knowledge. In this work, we investigate sufficient conditions for approximate abstractions that preserve near optimal behavior while reducing the complexity of the original task representation. We present theoretical guarantees of the quality of behaviors derived from four classes of approximate abstractions. Additionally, we empirically demonstrate reduction in task complexity and bounded loss of optimality of behavior as knowledge becomes more approximate.


%decreasing with degrees of incompleteness of knowledge in a variety of \acp{MDP}.
%
%reduction in task complexity at the cost of bounded increase in error of behavior.
%
%that the relationship between the degree of approximation and the degree of  achieved in a variety of example tasks, as well as the tradeoff between the degree of approximation and the optimality of behavior.

%that achieves high degrees of compression while preserving near optimal behavior
%In this work, we investigate a theory of approximate state abstraction that achieves high degrees of compression while preserving near optimal behavior.




% We fight exponentials with abstraction.