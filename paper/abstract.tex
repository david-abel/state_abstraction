The combinatorial explosion that plagues planning and \ac{RL} algorithms can be moderated using state abstraction. Prohibitively large task representations can be condensed such that essential information is preserved, and consequently, solutions are tractably computable. However, exact abstractions, which treat only fully-identical situations as equivalent, fail to present opportunities for abstraction in environments where no two situations are exactly alike. For these reasons, we investigate approximate state abstractions, which treat nearly-identical situations as equivalent. In this work, we characterize sufficient conditions for approximate abstractions that preserve near optimal behavior while reducing the complexity of the original task representation. We present theoretical guarantees of the quality of behaviors derived from four types of approximate abstractions. Additionally, we empirically demonstrate that approximate abstractions lead to reduction in task complexity and bounded loss of optimality of behavior in a variety of environments. % Environments could also be tasks/problems/etc

%as approximations become tighter, 
% as knowledge becomes more approximate.


% Old one:
% Abstraction plays a fundamental role in learning. Through abstraction, intelligent agents may reason about only the salient features of their environment while ignoring what is irrelevant, consequently enabling agents to solve considerably more complex problems. However, in natural environments, no two states are identical. This work characterizes the impact of combining ``sufficiently similar'' states in the context of planning and \ac{RL} in \acp{MDP}.


%Because of the difficulty of acquiring perfect knowledge, agents must often perform approximate abstraction on the basis of incomplete knowledge.




% There are a variety of attractive characteristics of approximate abstractions.

%decreasing with degrees of incompleteness of knowledge in a variety of \acp{MDP}.
%
%reduction in task complexity at the cost of bounded increase in error of behavior.
%
%that the relationship between the degree of approximation and the degree of  achieved in a variety of example tasks, as well as the tradeoff between the degree of approximation and the optimality of behavior.

%that achieves high degrees of compression while preserving near optimal behavior
%In this work, we investigate a theory of approximate state abstraction that achieves high degrees of compression while preserving near optimal behavior.




% We fight exponentials with abstraction.
