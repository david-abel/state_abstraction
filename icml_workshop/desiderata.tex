\section{Two Desiderata for Abstraction in RL}

We propose the following two desiderata on theories of abstraction in the context of planning or learning, given a Markov Decision Process, $M_G$:
\begin{enumerate}
\item \textsc{Compressibility}: A theory ought to give guidance on how to compress $M_G$ to be significantly smaller. \footnote{where smallness is understood as in \cite{littman1995complexity}, with respect to $|\mathcal{S}|$, $|\mathcal{A}|$, and the information content required to represent input-output pairs of $\mathcal{T}$ and $\mathcal{R}$.}
\item \textsc{Optimality}: A theory ought to provide a method for inducing an MDP $M_A$ such that solving $M_A$ basically solves $M_G$.
\end{enumerate}

% Defense of the desiderata.
Abstraction is aimed at reducing the dimensionality of an entity; this {\it reduction} is captured by \textsc{Compressibility}, while the {\it preservation} of critical information content of the entity is captured by \textsc{Optimality}. \dnote{Note about how to satisfy them?}.

Critically, either desiderata is trivial to satisfy on its own. Given an \ac{MDP}, $M_G$, we can satisfy \textsc{Compressibility} by replacing $S_G$ with a state space that consists of a single (arbitrary) state and a single (arbitrary action). Clearly, such a resulting \ac{MDP} satisfies our first desiderata - we are left with an MDP that is substantially smaller than the original MDP. Now consider \textsc{Optimality}. Again, consider the following uninteresting method; generate an MDP $M_A$, such that $M_A$ is identical to $M_G$. Thus, the optimal policy for $M_A$ is exactly the optimal policy for $M_G$, consequently satisfying \textsc{Optimality}.

The interesting (and difficult) case is satisfying {\it both} desiderata. To the best of our knowledge, no abstraction has been formally proven to satisfy the two - in the next section, we discuss prior methods of abstraction, and their statuses relative to these desiderata. After, we propose an extension of some prior work that suggests how such results may be obtained for temporal abstraction.